\section{Adding new PDFs}
\label{sec:newpdfs}

For advanced users who need esoteric functions, GooFit makes
it easy to write a new PDF class. There are two steps to the process:
\begin{itemize}
\item Write an evaluation method with the required signature,
using the provided index array to look up parameters and observables.
An example is shown in Listing~\ref{list:evalexample}.
\item Create a C++ class inheriting from \texttt{GooPdf}, in which
the constructor populates the index array through the \texttt{registerParameter}
method. Listing~\ref{list:indexexample} shows an example. 
\end{itemize}
That's it! Putting the new \texttt{FooPdf.cu} and \texttt{FooPdf.hh} files
in the \texttt{PDFs} directory of GooFit will cause them to be compiled with
the rest of the framework, and be available for use in the same way as the
pre-existing PDFs. 

\begin{listing}
\label{list:evalexample} \emph{Evaluation function for a Gaussian
PDF. Note the double indirection of the index-array lookups.
Here \texttt{fptype} indicates a floating-point number, by default \texttt{double}
precision.}

\begin{Verbatim}[commandchars=\\\$\#]
__device__ fptype device_Gaussian (fptype* evt, fptype* p, unsigned int* indices) {
  fptype x = evt[indices[2 + indices[0]]]; 
  fptype mean = p[indices[1]];
  fptype sigma = p[indices[2]];

  fptype ret = EXP(-0.5*(x-mean)*(x-mean)/(sigma*sigma));
  return ret; 
}

__device__ device_function_ptr ptr_to_Gaussian = device_Gaussian; 
\end{Verbatim}
\end{listing}

\begin{listing}
\label{list:indexexample} \emph{Populating the index array used in Listing~\ref{list:evalexample}}.

\begin{Verbatim}[commandchars=\\\$\#]
__host__ GaussianPdf::GaussianPdf (std::string n, Variable* _x, 
                                   Variable* mean, Variable* sigma) 
  : GooPdf(_x, n) 
{
  std::vector<unsigned int> pindices;
  pindices.push_back(registerParameter(mean));
  pindices.push_back(registerParameter(sigma));
  cudaMemcpyFromSymbol((void**) &host_fcn_ptr, ptr_to_Gaussian, sizeof(void*));
  initialise(pindices); 
}

\end{Verbatim}
\end{listing}
