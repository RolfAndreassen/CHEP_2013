\section{Combining functions}
\label{sec:combining}

To create arbitrarily-complex PDFs, the user can combine 
simple ones like the exponential and Gaussian in several
ways, the most common being addition and multiplication.
For example, a two-dimensional distribution, described by
a separate exponential in each of two variables $x$ and $y$, may be represented
as a \texttt{ProdPdf} object, as shown in Listing~\ref{list:prodpdf}.

\begin{listing}
\label{list:prodpdf} \emph{Product of two exponentials.}

\begin{Verbatim}[commandchars=\\\$\#]
int main (int argc, char** argv) {
  Variable* xvar = new Variable("xvar", 0, log(1+RAND_MAX/2)); 
  Variable* yvar = new Variable("yvar", 0, log(1+RAND_MAX/2)); 
  
  vector<Variable*> varList;
  varList.push_back(xvar);
  varList.push_back(yvar);
  UnbinnedDataSet data(varList);
  for (int i = 0; i < 100000; ++i) {
    xvar->value = xvar->upperlimit - log(1+rand()/2);
    yvar->value = yvar->upperlimit - log(1+rand()/2);
    data.addEvent(); 
  }
  
  Variable* alpha_x = new Variable("alpha_x", -2.4, 0.1, -10, 10);
  Variable* alpha_y = new Variable("alpha_y", -1.1, 0.1, -10, 10);
  vector<\fvtextcolor$red#$PdfBase*#> pdfList;
  pdfList.push_back(new ExpPdf("exp_x", xvar, alpha_x)); 
  pdfList.push_back(new ExpPdf("exp_y", yvar, alpha_y)); 

  \fvtextcolor$red#$ProdPdf*# product = new \fvtextcolor$red#$ProdPdf#("product", pdfList);
  product->setData(&data);

  FitManager fitter(product);
  fitter.fit();

  return 0;
}
\end{Verbatim}
\end{listing}

