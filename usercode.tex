\section{User-level code}
\label{sec:usercode}

The purpose of GooFit is to give users access to the parallelising power of CUDA
without requiring them to write CUDA code. At the most basic level, GooFit objects
representing PDFs, \texttt{GooPdf}s, can be created and combined in plain C++. Only
if a user needs to represent a function outside the existing GooFit classes does he
need to do any CUDA coding; Section~\ref{sec:newpdfs} shows how to create new PDF classes.
We intend, however, that this should be a rarity, and that the existing PDF classes
should cover all the most common cases. 

A GooFit program has four main components: 
\begin{itemize}
\item The PDF that models the physical process, represented by a \texttt{GooPdf} object.
\item The fit parameters with respect to which the likelihood is maximised,
represented by \texttt{Variable}s contained in the \texttt{GooPdf}. 
\item The data, gathered into a \texttt{DataSet} object containing one or more \texttt{Variable}s.
\item A \texttt{FitManager} object which forms the interface between MINUIT
(or, in principle, a different maximising algorithm) and the \texttt{GooPdf}.
\end{itemize}
Listing~\ref{list:expexp} shows a simple fit of an exponential function.
\begin{listing}
\label{list:expexp} 
\emph{Fit for unknown parameter $\alpha$ in $e^{\alpha x}$. GooFit classes
are shown in red, important operations in blue.}

\begin{Verbatim}[commandchars=\\\$\#]
int main (int argc, char** argv) {
  // Independent variable. 
  \fvtextcolor$red#$Variable*# xvar = new \fvtextcolor$red#$Variable#("xvar", 0, log(RAND_MAX)); 
  
  // Create data set
  \fvtextcolor$red#$UnbinnedDataSet# data(xvar);
  for (int i = 0; i < 100000; ++i) {
    // Generate toy event
    xvar->value = xvar->upperlimit - log(1+rand());
    if (xvar->value < 0) continue;
    // ...and add to data set. 
    \fvtextcolor$blue#$data.addEvent();# 
  }
  
  // Create fit parameter
  \fvtextcolor$red#$Variable*# alpha = new \fvtextcolor$red#$Variable#("alpha", -2, 0.1, -10, 10);
  // Create GooPdf object
  \fvtextcolor$red#$ExpPdf*# exppdf = new \fvtextcolor$red#$ExpPdf#("exppdf", xvar, alpha); 
  // Move data to GPU
  \fvtextcolor$blue#$exppdf->setData(&data);#

  \fvtextcolor$red#$FitManager# fitter(exppdf);
  \fvtextcolor$blue#$fitter.fit();# 

  return 0;
}
\end{Verbatim}
\end{listing}
